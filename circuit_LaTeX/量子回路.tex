\documentclass{jsarticle}
\usepackage[dvipdfmx]{graphicx}

\usepackage[T1]{fontenc}
\usepackage{lmodern}
\usepackage[cmex10]{amsmath}

\usepackage{qcircuit}
\usepackage{braket}


\title{レポートタイトル}

\author{学生番号XXX-XXXX アカリク太郎}
\date{\today}
\begin{document}
\maketitle

\section{サンプル}

\Qcircuit @C=1em @R=.7em {
  \lstick{\ket{0}} & \gate{H} & \ctrl{1} & \qw          & \qw      & \meter \\
  \lstick{\ket{0}} & \qw      & \targ    & \ctrl{1}     & \qw      & \meter \\
  \lstick{\ket{0}} & \qw      & \gate{H} & \control \qw & \gate{H} & \meter
}
\par


\section{交換回路}
単一制御NOTを作成する。\par
\Qcircuit @C=1em @R=2em {
  \lstick{\ket{A}} & \qw   & \qw  &   \ctrl{1}   & \qw  & \qw  &  \rstick{\ket{A}} \\
  \lstick{\ket{B}} & \qw   & \qw  &   \targ      & \qw  & \qw  &  \rstick{\ket{A \bigoplus B}}
}

\section{CNOT回路}



\section{制御制御NOTゲート}

\begin{thebibliography}{99}
\bibitem{1}伊藤公平, 工学者のための量子計算基礎の基礎, 「https://www.appi.keio.ac.jp/Itoh\_group/coursework/pdf/QCN1.pdf」, (2018/8/4アクセス).\par
\end{thebibliography}


\end{document}
